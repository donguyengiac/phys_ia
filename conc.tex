From the graph, we can see the best fit and manually fitted lines pass through all data points’ error bars. The horizontal error bars (0.001m each) are very small and cannot be seen on the graph compared to scale, so I didn’t attempt to fit the lines through them, only the vertical error bars. 
It can be observed that x2 and h are linearly correlated with proportionality constant of 0.0002028. From this, the experimental value of g can be determined to be 9.361.04 ms-2, which when compared with the accepted literature value of 9.81 ms-2, gives the percentage experimental error of 4.59\%. This experimental error is possibly due to the fact that energy is not fully conserved during the drop, and a little bit was lost as heat due to friction between the marble and the slope, but this was a small amount because both are relatively smooth surfaces. This can also be caused when the marble lightly bumps against the wall of the track when rolling down, dissipating energy. The graph does not pass through the origin due to a very small amount of systematic error. This could be due to a small zero error in my meter rule used to measure and mark out heights on the track, and also the thickness of the track. However, we can easily take these into account and minimize the systematic error to better confirm the relationship. Overall, the low values of experimental and systematic errors supports my Research Question and hypothesis, and shows that there is a linear relationship between varying initial height of a marble rolling down a ramp and the maximum compressed distance of a horizontal spring at the bottom upon collision.